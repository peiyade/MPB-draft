%
% Simple template for generating drafts of papers and articles
%
\documentclass[12pt,]{article}
\usepackage{authblk}
\usepackage{fullpage}
\usepackage{amssymb,amsmath}
\usepackage[utf8x]{inputenc}
\usepackage[T1]{fontenc}
\usepackage{siunitx}
\usepackage[version=3]{mhchem}

\usepackage{natbib}
\bibliographystyle{ametsoc2014}

\usepackage[left]{lineno}
\linenumbers

\usepackage{setspace}
\doublespacing

\usepackage[unicode=true]{hyperref}
\hypersetup{breaklinks=true,
            bookmarks=true,
            colorlinks=false,
            pdfborder={0 0 0}}
\urlstyle{same} % don't use a different (monospace) font for urls

\setcounter{secnumdepth}{5}

\usepackage{graphicx}
\graphicspath{{figures/}}
% Redefine \includegraphics so that, unless explicit options are
% given, the image width will not exceed the width or the height of the page.
% Images get their normal width if they fit onto the page, but
% are scaled down if they would overflow the margins.
\makeatletter
\def\ScaleWidthIfNeeded{%
 \ifdim\Gin@nat@width>\linewidth
    \linewidth
  \else
    \Gin@nat@width
  \fi
}
\def\ScaleHeightIfNeeded{%
  \ifdim\Gin@nat@height>0.9\textheight
    0.9\textheight
  \else
    \Gin@nat@width
  \fi
}
\makeatother
\setkeys{Gin}{width=\ScaleWidthIfNeeded,height=\ScaleHeightIfNeeded,keepaspectratio}%

%%%%%%%%%%%%%%%%%%%%%%%%%%%%%%%%%%%%%%%%%%%%%%%%%%%%%%%%%%%%%%%%%%%%%%%%%%%%%%

\title{draft}

\date{\today}

%%%%%%%%%%%%%%%%%%%%%%%%%%%%%%%%%%%%%%%%%%%%%%%%%%%%%%%%%%%%%%%%%%%%%%%%%%%%%%

\begin{document}

\maketitle

Considering variable separation functional in MPB with porous medium equations

\begin{equation}
\left\{\begin{array}{l}
u_{t}=\Delta u^m-\nabla \cdot(u \nabla v) \\
v_{t}=\Delta v-v w \\
w_{t}=-\delta w+u.
\end{array}\right.
\end{equation}

Define a variable separation functional $F(\cdot,\cdot)$ 
\begin{equation}
F(u,v) = \int u^{p} \varphi(v)
\end{equation}
for some $\varphi$ will be determined later.

Therefore, 
\begin{equation}
\begin{aligned}
\frac{\mathrm{d}}{\mathrm{d} t} \int u^{p} \varphi(v)=&p\int u^{p-1}\varphi(v)\partial_t u  + \int u^p\varphi'(v)\partial_tv\\
=&p\int u^{p-1}
\end{aligned}
\end{equation}


% \newpage\clearpage

% \renewcommand\refname{References}
% % This is the default "example" bibtex file which ships with the distribution. Be sure to
% % comment out the following line if you want to disable citing all bibtex entries by
% % default.
% \bibliography{xampl.bib}
% \newpage

\end{document}

%%%%%%%%%%%%%%%%%%%%%%%%%%%%%%%%%%%%%%%%%%%%%%%%%%%%%%%%%%%%%%%%%%%%%%%%%%%%%%