%
% Simple template for generating drafts of papers and articles
%
\documentclass[12pt,]{article}
\usepackage{authblk}
\usepackage{fullpage}
\usepackage{amssymb,amsmath}
\usepackage[utf8x]{inputenc}
\usepackage[T1]{fontenc}
\usepackage{siunitx}
\usepackage[version=3]{mhchem}

\usepackage{natbib}
\bibliographystyle{ametsoc2014}

\usepackage[left]{lineno}
\linenumbers

\usepackage{setspace}
\doublespacing

\usepackage[unicode=true]{hyperref}
\hypersetup{breaklinks=true,
            % bookmarks=true,
            colorlinks=false,
            pdfborder={0 0 0}}
\urlstyle{same} % don't use a different (monospace) font for urls

\setcounter{secnumdepth}{5}

\usepackage{graphicx}
\graphicspath{{figures/}}
% Redefine \includegraphics so that, unless explicit options are
% given, the image width will not exceed the width or the height of the page.
% Images get their normal width if they fit onto the page, but
% are scaled down if they would overflow the margins.
\makeatletter
\def\ScaleWidthIfNeeded{%
 \ifdim\Gin@nat@width>\linewidth
    \linewidth
  \else
    \Gin@nat@width
  \fi
}
\def\ScaleHeightIfNeeded{%
  \ifdim\Gin@nat@height>0.9\textheight
    0.9\textheight
  \else
    \Gin@nat@width
  \fi
}
\makeatother
\setkeys{Gin}{width=\ScaleWidthIfNeeded,height=\ScaleHeightIfNeeded,keepaspectratio}%

%%%%%%%%%%%%%%%%%%%%%%%%%%%%%%%%%%%%%%%%%%%%%%%%%%%%%%%%%%%%%%%%%%%%%%%%%%%%%%

\title{draft}

\date{\today}

%%%%%%%%%%%%%%%%%%%%%%%%%%%%%%%%%%%%%%%%%%%%%%%%%%%%%%%%%%%%%%%%%%%%%%%%%%%%%%

\begin{document}

% \maketitle

Considering variable separation functional in MPB with porous medium equations

\begin{equation}
\left\{\begin{array}{l}
u_{t}=\Delta u^m-\nabla \cdot(u \nabla v) \\
v_{t}=\Delta v-v w \\
w_{t}=-\delta w+u.
\end{array}\right.
\end{equation}

Define a variable separation functional $F(\cdot,\cdot)$ 
\begin{equation}
F(u,v) = \int u^{p} \varphi(v)
\end{equation}
for some $\varphi$ will be determined later.

Therefore, 
\begin{equation}
\begin{aligned}
\frac{\sigma_{p}}{p}\frac{\mathrm{d}}{\mathrm{d} t} \int u^{p} \varphi(v)=\sigma_{p}&\int u^{p-1}\varphi(v)\partial_t u  + \frac{\sigma_{p}}{p}\int u^p\varphi'(v)\partial_tv\\
  =&\sigma_{p}\int u^{p-1}\varphi(v)\Delta u^m- \sigma_{p}\int u^{p-1}\varphi(v)\nabla\cdot(u\nabla v)\\
  &+ \frac{\sigma_{p}}{p}\int u^p\varphi'(v)\Delta v -\frac{\sigma_{p}}{p} \int u^p\varphi'(v)vw\\
  :=& I_1 + I_2 + I_3 + I_4.
\end{aligned}
\end{equation}
\begin{equation}
  \begin{aligned}
    I_1=&-\sigma_{p}\int \nabla(u^{p-1}\varphi(v))\cdot\nabla u^m\\
    =&-|(p-1)m|\int \varphi(v)u^{p-2}\nabla u \cdot u^{m-1}\nabla u - m\sigma_{p}\int u^{p-1}\varphi'(v)\nabla v \cdot u ^{m-1}\nabla u \\
    =&-|(p-1)m|\int u^{p+m-3}\varphi(v)^2|\nabla u| ^2 - m\sigma_{p} \int u^{p+m-2}\varphi'(v)\nabla u \cdot \nabla v\\
    =&-|(p-1)m|\int u^{p+m-3}\varphi(v)^2|\nabla u| ^2 + J
  \end{aligned}
\end{equation}
\begin{equation}
  \begin{aligned}
    J =& m\sigma_{p}\int u^{\frac{p+m-3}{2}}\nabla u\cdot\varphi'(v)u^{\frac{p+m-1}{2}}\nabla v\\
    =&\frac{|(p-1)m|}{4} \int u^{p+m-3}\varphi(v)^2|\nabla u|^2 + \frac{m}{|(p-1)|}\int \frac{\varphi'(v)^2}{\varphi(v)^2}u^{p+m-1}|\nabla v|^2
  \end{aligned}
\end{equation}
\begin{equation}
  \begin{aligned}
    I_2 =& \sigma_{p}\int \nabla(u^{p-1}\varphi(v))\cdot u\nabla v\\
    =& \sigma_{p}\int u^p\varphi'(v)|\nabla v|^2 + \sigma_{p}(p-1)\int u^{p-1}\varphi(v)\nabla u\cdot \nabla v\\
    = & \sigma_{p}\int u^p\varphi'(v)|\nabla v|^2 + \sigma_{p}(p-1)\int u^{\frac{p+m-3}{2}}\varphi(v)\nabla u \cdot u^{\frac{p-m+1}{2}} \nabla v\\
    \leqslant & \sigma_{p}\int u^p\varphi'(v)|\nabla v|^2 + \frac{|(p-1)m|}{4} \int u^{p+m-3}\varphi^2(v)|\nabla u|^{2} + \frac{|(p-1)|}{m}\int u^{p-m+1}|\nabla v|^2
  \end{aligned}
\end{equation}

\begin{equation}
  \begin{aligned}
    I_3=&\frac{\sigma_{p}}{p}\int u^p\varphi'(v)\Delta v = -\frac{\sigma_{p}}{p}\int \nabla(u^p\varphi'(c))\cdot\nabla v\\
    =&-\sigma_{p}\int u^{p-1}\varphi'(v)\cdot \nabla u\cdot\nabla v -\frac{\sigma_{p}}{p} \int u^p\varphi''(v)|\nabla v|^2\\
    \leqslant & \frac{|(p-1)m|}{4}\int u^{p+m-3}\varphi^2(v)|\nabla u |^{2}+\frac{1}{|(p-1)m|}\int u^{p-m+1}\frac{\varphi'(v)^2}{\varphi(v)^2}|\nabla v|^2\\
    &-\frac{\sigma_{p}}{p} \int u^p\varphi''(v)|\nabla v|^2
  \end{aligned}
\end{equation}
In conclusion, we have
\begin{equation}
  \begin{aligned}
    &\quad\frac{\sigma_{p}}{p}\frac{\mathrm{d}}{\mathrm{d} t} \int u^{p} \varphi(v) +  \frac{|(p-1)m|}{4}\int u^{p+m-3}\varphi(v)^2|\nabla u| ^2 \\
    &\leqslant \int u^{p-m+1}\left(\frac{|p-1|}{m}+\frac{1}{|(p-1)m|}\frac{\varphi'(v)^2}{\varphi(v)^2}\right)|\nabla v|^2\\
    &\quad + \int u^p\left(-\frac{\sigma_{p}}{p}\varphi''(v)+\sigma_{p}\varphi'(v)\right)|\nabla v|^2\\
    &\quad +\frac{m}{|(p-1)|}\int \frac{\varphi'(v)^2}{\varphi(v)^2}u^{p+m-1}|\nabla v|^2\\
    &\quad - \frac{\sigma_{p}}{p}\int u^p\varphi'(v)vw
  \end{aligned}
\end{equation}


% \newpage\clearpage

% \renewcommand\refname{References}
% % This is the default "example" bibtex file which ships with the distribution. Be sure to
% % comment out the following line if you want to disable citing all bibtex entries by
% % default.
% \bibliography{xampl.bib}
% \newpage

\end{document}

%%%%%%%%%%%%%%%%%%%%%%%%%%%%%%%%%%%%%%%%%%%%%%%%%%%%%%%%%%%%%%%%%%%%%%%%%%%%%%